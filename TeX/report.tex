\documentclass[titlepage,11pt]{article}
\usepackage[utf8]{inputenc}
\usepackage{fullpage}
\usepackage{indentfirst}
\usepackage[per-mode=symbol]{siunitx}
\usepackage{listings}
\usepackage{graphicx}
\usepackage{color}
\usepackage{amsmath}
\usepackage{array}
\usepackage[hidelinks]{hyperref}
\usepackage[format=plain,font=it]{caption}
\usepackage{subcaption}
\usepackage{standalone}
\usepackage[nottoc]{tocbibind}
\usepackage{scrextend}
\usepackage[margin=1in]{geometry}

\definecolor{dkgreen}{rgb}{0,0.6,0}
\definecolor{gray}{rgb}{0.5,0.5,0.5}
\definecolor{mauve}{rgb}{0.58,0,0.82}

\lstset{frame=tb,
  language=Java,
  aboveskip=3mm,
  belowskip=3mm,
  showstringspaces=false,
  columns=flexible,
  basicstyle={\small\ttfamily},
  numbers=none,
  numberstyle=\tiny\color{gray},
  keywordstyle=\color{blue},
  commentstyle=\color{dkgreen},
  stringstyle=\color{mauve},
  breaklines=true,
  breakatwhitespace=true,
  tabsize=3
}

\def\equationautorefname~#1\null{%
  Equation~(#1)\null
}

\def\sectionautorefname~#1\null{%
  Section~#1\null
}

\def\subsectionautorefname~#1\null{%
  Subsection~#1\null
}

% Custom commands
\newcommand\numberthis{\addtocounter{equation}{1}\tag{\theequation}}
\newcolumntype{P}[1]{>{\centering\arraybackslash}p{#1}}
\newcommand{\code}[1]{\texttt{#1}}

\title{\textbf{\uppercase{Prometheus AI Phase 1}}}
\author{
Sean Stappas \\
260639512
}
\date{April 10, 2017}

\begin{document}
\lstset{language=Java} 
\sloppy

\maketitle

\section*{Abstract}
% The abstract is an executive summary of your project/thesis. It should provide an overview of your project/thesis by addressing the following questions. a. What is the motivation for the project/thesis? b. What are the goals of the project/thesis? c. What was achieved this semester? d. What methods were used to make those achievements? The abstract should be between 200 and 250 words long.

\section*{Acknowledgments}
% If applicable, you may acknowledge people here who contributed to the project in some way but are not listed on the title page. (For example, if you received advice, data, or supervision from graduate students or others.)

\clearpage
\tableofcontents
\clearpage

\twocolumn

\section*{Abbreviations}
% List of abbreviations and/or notation used in the report.

\begin{labeling}{abbreviations}
\item [NN] Neural Network
\item [KNN] Knowledge Node Network
\item [ES] Expert System
\item [META] Meta Reasoner
\item [OOP] Object-Oriented Programming
\end{labeling}

\section{Introduction} \label{sec:intro}
% In this section you should introduce (at a high level) the overall theme of the project, and state clearly what are the goals you are trying to achieve.  This section should also clearly convey why this project is important, what is the potential impact (applications, etc).

The goal of this project is to create an artificial intelligence system to control multiple robots. Applications for this type of system include robots in hazardous environments, such as in outer space (Mars, Moon, etc.), in nuclear plants after a nuclear disaster, and in military zones.

The structure of the system is inspired from the functionality of the human brain, and is composed of the following four layers (in order of increasing abstraction): the Neural Network (NN), the Knowledge Node Network (KNN), the Expert System (ES), and the Meta Reasoner (META). These will be described in \autoref{sec:background}.

\section{Background} \label{sec:background}
% In this section you should summarize the theory and background that you had to learn, and that you believe are necessary for the reader to understand the rest of the report.

\subsection{Neural Network}

The NN layer consists of a network of neurons with a similar structure to neurons in the human brain. In the context of this project, it is the interface between the robots' sensors and the rest of the AI system. Informational tags are generated by this layer based on observations that the robots make in their environment. These tags are passed

\subsection{Knowledge Node Network}

The KNN layer represents memory in the human brain. It takes in the tags provided by the NN and outputs tags based on its knowledge of the environment. These output tags can be simple facts, such as ``I see a wall'', or can be recommendations for future actions such ``Turn left''. These tags are passed on to the next layer, the Expert System (ES).

\subsection{Expert System}

The ES layer is a decision maker based on pure facts. It is not aware of its current reality, or any context. It simply has a list of known Facts, in the form of tags, and outputs tags based on known relationships, called Rules. These output tags are passed on to the final layer, the Meta Reasoner (META).

\subsection{Meta Reasoner}

The META layer represents high-level reasoning in human brains. It is aware of its environment and context, and makes decisions based on what it believes to be right. It is paranoid, and constantly checks whether the tags reported by the rest of the AI system make sense based on its expected view of the world. Once it makes a decision, it sends a command to the actuators of the robots to decide how to move.

\section{Problem}
% In this section you should describe the problem or system you are addressing in detail. Describe the project requirements and constraints.

The assigned task was to construct two out of the four layers listed in \autoref{sec:background}: the Expert System (ES), and the Knowledge Node Network (KNN). The other two layers are to be completed by another Honours Thesis student.

As a deliverable for the end of this first semester, it was required to complete a prototype of these two layers, with basic versions of the functionality described in \autoref{sec:background}. This was to be done entirely in Java.

\section{Design}
% In this section you must describe any design work that was already completed. Discuss any design decisions that have been made, and describe the process that was followed to make these decisions. Also discuss any results that have already been obtained this semester.



\subsection{Efficiency (Space \& Time)}

A very important consideration when designing the system is speed. Since the robots may have to react very quickly to stimulus in the environment, the reasoning in the AI must be as fast as possible. This is especially true in the hazardous environments for which this system could be useful for, as specified in \autoref{sec:intro}. An example of a design choice that was made to improve speed is the use of Java Sets for most of the collections in the ES and KNN layers. The original specifications mentioned using ArrayLists, but, since there is no specific iteration order necessary for most operations in the ES and KNN, these collections were changed to Sets.

When designing every algorithm in the system, the first criterion in mind is speed. Indeed, for the \code{think()} methods of the ES and KNN, it was important to think about the right way to leverage all the available data structures to maximize speed.

\subsection{Object Oriented Design}

Another important choice is to leverage object-oriented design as much as possible. Object-oriented programming (OOP) allows extensive planning before even beginning to write code, which can identify any flaws in the initial design. It also allows the code to be very clean and reusable. Since Java is the programming language chosen for the project, OOP is also the natural way to proceed. To follow OOP, all the structures described in \autoref{sec:background} will be designed around Java classes and the methods associated with those classes. OOP principles such as polymorphism and encapsulation will be followed closely.

For example, the ``tags'' described in \autoref{sec:background} need to be as general as possible. A natural choice for this structure would be a Java String, which would be relatively simple to pass around the system. However, these tags represent various concepts; each tag can either be a Fact, a Recommendation, or a Rule. If implemented as Strings, the tags would have to be encoded on creation to represent each concept and decoded on use to retrieve the important information. This seems like a bad use of the OOP principles of Java. For this reason, the tags are instead implemented using a Tag Java class, with Recommendation, Fact, and Rule subclasses. This should also make manipulating the Tags faster, while incurring a slight memory overhead. To store these Tags in a database, they can be converted to JSON format. On read from the database, they can be easily decoded.

\subsubsection{Abstraction}

One important design aspect is that the system should be as abstract as possible, while still performing its desired task. For instance, the system should be general enough to perform under simulations, as well as in real-life environments. It should also ideally be able to perform in vastly different environments, with different tasks.

One example of making use of abstraction is the choice to create an interface for both the ES and KNN, since they share some functionality, like the ability to think.

\subsubsection{Encapsulation}

Each layer of the system has unique, localized functionality that does not need to be visible from the rest of the system. For instance, the intricacies of the \code{think()} method in the ES and KNN layers do not need to be known to the rest of the system.

\subsubsection{Polymorphism}

Many methods in the system can have different functionality depending on context. For instance, the \code{think()} method in the KNN and ES layers can take an optional number of think cycles to specify a threshold for quiescence.

\subsubsection{Inheritance}

\subsection{Implementation}

The details of the specific design choices for the Expert System and Knowledge Node Network will now be discussed.

\subsubsection{Expert System}

The Expert System layer is based around the \code{ExpertSystem} Java class.

\subsubsection{Knowledge Node Network}

The Knowledge Node Network layer is based around the \code{KnowledgeNodeNetwork} Java class.

\subsection{Testing}

\subsubsection{Unit Tests}

Unit tests were generated for the KNN and ES.

\subsubsection{Integration Tests}

Integration tests were made to test the combined functionality of the KNN and ES.

\subsection{Documentation}

Proper documentation of the source code is very important. This is to ensure that anyone wanting to work with the code which was designed has an easy time doing so. Also, for anyone wanting to understand how the system works, documentation is essential. This documentation was achieved through Javadoc and UML diagrams.

\subsubsection{Javadoc}

Extensive Javadoc was generated for the entire code base. This can be found here: \url{http://cs.mcgill.ca/~sstapp/prometheus/index.html}.

\subsubsection{UML}

UML diagrams of every package in the code base were created.

\section{Plan for Next Semester}
% In this section you should describe a plan for next semester. What tasks remain to be done, and what is the timeline for accomplishing these tasks? Furthermore you can describe how you plan to test your design in order to make sure it meets the desired specifications. Also outline the design decisions that are required in order to make sure that the final product can be tested.
% TODO: Maybe include schedule

The plan for next semester is to finalize the two layers that were started this semester, and to test them on the robots available in Prof. Vybihal's lab.

\subsection{Finalization}

\subsubsection{Knowledge Node Network}

Many complex features of the KNN layer are still to be implemented, such as backwards and lambda thinking, confidence, aging...

\subsubsection{Expert System}

Some features of the ES layer are still to be implemented, such as more complex Fact checking.

\subsection{Integration}

One very important task left to be done is to integrate the two layers described in this report (ES and KNN) with the other layers developed separately (NN and META). Ideally, the layers should be able to work together, but there will surely be some conflicts at the interface of the layers. These will have to be resolved when the time comes.

\subsection{Testing}

\subsubsection{Simulation}

Once all the layers are functioning together, they can be tested. The first and easiest way to test would be in a simulated environment. One simulator that may be used is Simbad, which is a Java 3D robot simulator \cite{simbad}. This can allow for some early debugging and fixes.

\subsubsection{Physical}

One the simulation testing is completed and working properly, the system can be tested in the lab. Prof. Vybihal's lab has multiple robots with ultrasonic sensors, and these will be the test subjects of this phase.

\section{Impact on Society and the Environment}
% In one or two pages, discuss the environmental and/or social impact of your project. Your analysis should include your work at McGill as part of this project, however, the main focus should be on the product/system you are designing (e.g., the cost/benefit/risk of manufacturing it, the cost/benefit/risk for consumers using it) or the problem your thesis is addressing (e.g., how will solving the problem influence/affect/benefit society and the environment). Particular emphasis should be given to:


\subsection{Use of Non-renewable Resources}
% Consider all stages of the product from design, manufacturing, distribution, use by consumers, and disposal/recycling.

As a purely software-oriented project, there are no physical materials needed to construct this system.

\subsection{Environmental Benefits}
% Benefits to the environment, comparisons with more polluting technologies etc.

The system could be used as a tool to control robots in dangerous environments such as nuclear plants after a radioactive disaster. Indeed, the system could coordinate robots to help contain the damage faster than humans could, thus limiting the risk on the environment.

\subsection{Safety and Risk}
% To you undertaking the project, as well as to any potential users of the solution once it is finished.

It is critical that, once this system is completed, it is used in an ethical way, and for the right purposes.

One example of use that may cause ethical concern is in a military setting, where an AI system like the one described here could be used in a battlefield in place of soldiers.

In the very long term, there are concerns with the possibility that an AI system might achieve intelligence and awareness close to a human. If such an AI were to obtain ``consciousness'' in its own way, should that entity be entitled to its own rights, like humans do?

\subsection{Benefits to Society}
% Quality of life, economic benefits, etc.

Going back to the example use case in a radioactive disaster, the system could be used to send robots in an area that would otherwise be very dangerous for humans. This would therefore help prevent the unnecessary loss of life in cleaning up these leaks.

\section{Conclusion}
% Use this section to summarize what was accomplished in this semester, provide a summary of the next steps and share any insight you have learned.

This semester, prototypes of the Expert System (ES) and Knowledge Node Network (KNN) layers of the AI model were completed in Java. These prototypes were tested individually and together, with positive results.

The main goal for next semester is to finalize the entire system, implementing more complex features that were omitted for this prototype stage.

\clearpage

\bibliography{readings}{}
\bibliographystyle{IEEEtran}

\end{document}